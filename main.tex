% !TeX encoding = UTF-8
% !TeX program = xelatex
% !TeX spellcheck = en_US

\documentclass{cjc}

\usepackage{booktabs}
\usepackage{siunitx}
\usepackage{hyperref}



\begin{document}

\cjcsetup{
  title        = {题目},
  title*       = {Title},
  authors  = {作者,作者},
  email = {邮箱,邮箱},
  authors* = {name, name},
  email* = {Email},
  affiliations = {
    aff1 = {
      name  = {单位全名 部门(系)全名, 市(或直辖市) 国家名 邮政编码},
      name* = {Department of ****, University, City ZipCode, Country},
    },
    aff2 = {
      name  = {单位全名 部门(系)全名, 市(或直辖市) 国家名 邮政编码},
      name* = {Department of ****, University, City ZipCode, Country},
    },
    aff3 = {
      name  = {单位全名 部门(系)全名, 市(或直辖市) 国家名 邮政编码},
      name* = {Department of ****, University, City ZipCode, Country},
    },
  },
  abstract     = {
    中文摘要内容置于此处(英文摘要中要有这些内容),字体为小5号宋体。
    摘要贡献部分,要有数据支持,不要出现“...大大提高”、“...显著改善”等描述,
    正确的描述是“比…提高X\%”、 “在…上改善X\%”。
  },
  % 中文关键字与英文关键字对应且一致,应有5-7个关键词
  keywords     = {关键词, 关键词, 关键词, 关键词},
  abstract*    = {Abstract},
  % 中文关键字与英文关键字对应且一致,不要用英文缩写
  keywords*    = {key word, key word, key word, key word},
}

\maketitle



\section{一级标题}

对投稿的基本要求:

(1)研究性论文主体应包括引言(重点论述研究的科学问题、意义、解决思路、价值、
贡献等)、相关工作(为与引言部分独立的一个章节)、主要成果论述、关键实现技术、
验证(对比实验或理论证明)、结论(结束语)等内容;系统实现或实验应有关键点的详细论述,以便读者能够重复实现论文所述成果。实验应有具体的实验环境设置、全面细致的数据对比分析。

(2)综述应包括引言、问题与挑战、研究现状分析、未来研究方向、结论等内容。以分析、对比为主,避免堆砌文献或一般性介绍、叙述。

(3)定理证明、公式推导、大篇幅的数学论述、原始数据,放到论文最后的附录中。

稿件提交时的基本要求:

(1)本模板中要求的各项内容正确齐全,无遗漏;

(2)语句通顺,无中文、英文语法错误,易于阅读理解,符号使用正确,图、表清晰无误;

(3)在学术、技术上,论文内容正确无误,各项内容确定。

\subsection{二级标题}

\subsubsection{三级标题}

正文部分, 字体为5号宋体。

文件排版采用 TeX Live。

正文文字要求语句通顺,无语法错误,结构合理,条理清楚,不影响审稿人、读者阅读理解全文内容。以下几类问题请作者们特别注意:

1)文章题目应明确反映文章的思想和方法;文字流畅,表述清楚;

2)中文文字、英文表达无语法错误;

3)公式中无符号、表达式的疏漏,没有同一个符号表示两种意思的情况;

4)数学中使用的符号、函数名用斜体;

5)使用的量符合法定计量单位标准;

6)矢量为黑体,标量为白体;

7)变量或表示变化的量用斜体;

8)图表规范,量、线、序无误,位置正确(图表必须在正文中有所表述后出现,即…如图1所示)(注意纵、横坐标应有坐标名称和刻度值)。

9)列出的参考文献必须在文中按顺序引用,即参考文献顺序与引用顺序一致,各项信息齐全(格式见参考文献部分);

10)首次出现的缩写需写明全称,首次出现的符号需作出解释。

11)图的图例说明、坐标说明全部用中文或量符号。

12)图应为矢量图。

13)表中表头文字采用中文。

14)公式尺寸:

标准:10.5磅

下标/上标:5.8磅

次下标/上标:4.5磅

符号:16磅

次符号:10.5磅

15)组合单位采用标准格式,如:“pJ/bit/m4”应为“\si{pJ/(bit.m^4)}”


\begin{theorem}
  定理内容。
  “定义”、“假设”、“公理”、“引理”等的排版格式与此相同,详细定理证明、公式可放在附录中。
\end{theorem}

\begin{proof}
  证明过程.
\end{proof}

\begin{figure}[htb]
  \centering
  \includegraphics[width=\linewidth]{example-fig.pdf}
  \caption{图片说明 *字体为小 5 号,图片应为黑白图,图中的子图要有子图说明*}
\end{figure}

\begin{table}[htb]
  \centering
  \caption{表说明 *表说明采用黑体*}
  \small
  \begin{tabular}{cc}
    \toprule
    示例表格 & 第一行为表头,表头要有内容 \\
    \midrule
    & \\
    \midrule
    & \\
    \bottomrule
  \end{tabular}
\end{table}

% \begin{procedure}
%   定理内容。
%   “定义”、“假设”、“公理”、“引理”等的排版格式与此相同,详细定理证明、公式可放在附录中。
% \end{procedure}

% \begin{algorithm}
%   定理内容。
%   “定义”、“假设”、“公理”、“引理”等的排版格式与此相同,详细定理证明、公式可放在附录中。
% \end{algorithm}



\section*{致谢}

致谢内容。


\nocite{*}

\bibliographystyle{cjc}
\bibliography{refs}



\appendix

\section{}

附录内容置于此处,字体为小5号宋体。附录内容包括:详细的定理证明、公式推导、原始数据等



% \begin{background}
% *论文背景介绍为英文,字体为小5号Times New Roman体*

% 论文后面为400单词左右的英文背景介绍。介绍的内容包括:

% 本文研究的问题属于哪一个领域的什么问题。该类问题目前国际上解决到什么程度。

% 本文将问题解决到什么程度。

% 课题所属的项目。

% 项目的意义。

% 本研究群体以往在这个方向上的研究成果。

% 本文的成果是解决大课题中的哪一部分,如果涉及863/973以及其项目、基金、研究计划,注意这些项目的英文名称应书写正确。
% \end{background}

\end{document}
